\usepackage{xparse}
\usepackage{pgfkeys}

\pgfkeys{
 /a/.is family, /a,
 % Here are the options that a user can pass
 default/.style = 
  {width = \textwidth, height = \baselineskip,
   position = center, inner position = center},
 width/.estore in = \myparboxWidth,
 height/.estore in = \myparboxHeight,
 position/.style = {positions/#1/.get = \myparboxPosition},
 inner position/.style = {positions/#1/.get = \myparboxInnerPos},
 % Here is the dictionary for positions.
 positions/.cd,
  top/.initial = t,
  center/.initial = c,
  bottom/.initial = b,
  stretch/.initial = s,
}

% We process the options first, then pass them to `\parbox` in the form of macros.
\newcommand\myparbox[2][]{%
 \pgfkeys{/a, default, #1}%
 \parbox[\myparboxPosition][\myparboxHeight]
        [\myparboxInnerPos]{\myparboxWidth}{#2}
}

\pgfkeys{
/report/.is family,
/report,
title/.estore in = \rtitle,
subtitle/.estore in = \rstitle,
theme/.estore in = \rtheme,
group/.estore in = \rgroupnum,
period/.estore in = \rperiod,
supervisor/.estore in = \rsupervisor,
study/.estore in = \rstudy,
authors/.estore in = \rauthors,
}

\newcommand{\SetupReport}[1]{
    \pgfkeys{/report,#1} % Don't add space between , and # it will break everything
    \title{\rtitle}
    \author{\GetAuthorList[L]}
    \date{\now}
    \hypersetup{
        pdftitle={\rtitle},
        pdfauthor={\rauthors},
        pdfsubject={\rtheme}
    }
}

\ExplSyntaxOn
% Command for converting a comma separated list
% to a line separated list
\NewDocumentCommand{\listCTL}{ m }{
    \clist_map_inline:nn { #1 } { ##1 \\ }
}
\ExplSyntaxOff

\newcommand{\GetAuthorList}[1][C]{
    \if C#1%
        \rauthors, test
    \else%
        \expandafter\listCTL\expandafter{\rauthors}
    \fi
}

%%%%%%%%%%%%%%%%%%%%%%%%%%%%%%%%%%%%%%%%%%%%%%%%
% Macros for the titlepage
% From : https://github.com/jkjaer/aauLatexTemplates/blob/master/aauReportTemplate/setup/macros.tex
%%%%%%%%%%%%%%%%%%%%%%%%%%%%%%%%%%%%%%%%%%%%%%%%
%Creates the aau titlepage
\newcommand{\aautitlepage}[3]{%
  {
    %set up various length
    \ifx\titlepageleftcolumnwidth\undefined
      \newlength{\titlepageleftcolumnwidth}
      \newlength{\titlepagerightcolumnwidth}
    \fi
    \setlength{\titlepageleftcolumnwidth}{0.5\textwidth-\tabcolsep}
    \setlength{\titlepagerightcolumnwidth}{\textwidth-2\tabcolsep-\titlepageleftcolumnwidth}
    %create title page
    \thispagestyle{empty}
    \noindent%
    \begin{tabular}{@{}ll@{}}
      \parbox{\titlepageleftcolumnwidth}{
        \includegraphics[width=\titlepageleftcolumnwidth]{media/AAUGraphics/aau_logo_en}
      } &
      \parbox{\titlepagerightcolumnwidth}{\raggedleft\sf\small
        #2
      }\bigskip\\
       #1 &
      \parbox[t]{\titlepagerightcolumnwidth}{%
      \textbf{Abstract:}\bigskip\par
        \fbox{\parbox{\titlepagerightcolumnwidth-2\fboxsep-2\fboxrule}{%
          #3
        }}
      }\\
    \end{tabular}
    \vfill
    \noindent{\footnotesize\emph{The content of this report is freely available, but publication (with reference) may only be pursued due to agreement with the author.}}
    \clearpage
  }
}

%Create english project info
\newcommand{\englishprojectinfo}[8]{%
  \parbox[t]{\titlepageleftcolumnwidth}{
    \textbf{Title:}\\ #1\bigskip\par
    \textbf{Theme:}\\ #2\bigskip\par
    \textbf{Project Period:}\\ #3\bigskip\par
    \textbf{Project Group:}\\ #4\bigskip\par
    \textbf{Participant(s):}\\ #5\bigskip\par
    \textbf{Supervisor(s):}\\ #6\bigskip\par
    \textbf{Copies:} #7\bigskip\par
    \textbf{Page Numbers:} \pageref{LastPage}\bigskip\par
    \textbf{Date of Completion:}\\ #8
  }
}

%Create danish project info
\newcommand{\danishprojectinfo}[8]{%
  \parbox[t]{\titlepageleftcolumnwidth}{
    \textbf{Titel:}\\ #1\bigskip\par
    \textbf{Tema:}\\ #2\bigskip\par
    \textbf{Projektperiode:}\\ #3\bigskip\par
    \textbf{Projektgruppe:}\\ #4\bigskip\par
    \textbf{Deltager(e):}\\ #5\bigskip\par
    \textbf{Vejleder(e):}\\ #6\bigskip\par
    \textbf{Oplagstal:} #7\bigskip\par
    \textbf{Sidetal:} \pageref{LastPage}\bigskip\par
    \textbf{Afleveringsdato:}\\ #8
  }
}

%%%%%%%%%%%%%%%%%%%%%%%%%%%%%%%%%%%%%%%%%%%%%%%%
% An example environment
%%%%%%%%%%%%%%%%%%%%%%%%%%%%%%%%%%%%%%%%%%%%%%%%
\theoremheaderfont{\normalfont\bfseries}
\theorembodyfont{\normalfont}
\theoremstyle{break}
\def\theoremframecommand{{\color{gray!50}\vrule width 5pt \hspace{5pt}}}
\newshadedtheorem{exa}{Example}[chapter]
\newenvironment{example}[1]{%
		\begin{exa}[#1]
}{%
		\end{exa}
}